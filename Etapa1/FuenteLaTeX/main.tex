\documentclass[titlepage]{article}
\usepackage[spanish]{babel}
\usepackage[utf8]{inputenc}
\usepackage{amsmath,amssymb,amsfonts,latexsym}
\setlength{\columnsep}{7mm}
\usepackage{makecell}
\usepackage[T1]{fontenc}
\usepackage[utf8]{inputenc}
\usepackage{multicol}
\usepackage{longtable}
\usepackage{graphicx, tabularx}
\usepackage{float}
\usepackage{algorithmic}
\usepackage[dvipsnames]{xcolor}
\usepackage{booktabs}
\usepackage{tikz,tkz-tab}
\usepackage[letterpaper, margin=1in]{geometry}
\renewcommand{\arraystretch}{1.2}
\usepackage{authblk}
\newcommand\tab[1][1cm]{\hspace*{#1}}

\usepackage{changepage}
\newlength{\offsetpage}
\setlength{\offsetpage}{1.5cm}
\newenvironment{widepage}{\begin{adjustwidth}{-\offsetpage}{-\offsetpage}%
    \addtolength{\textwidth}{2\offsetpage}}%
{\end{adjustwidth}}
\usepackage{listings}

\title{PC4 Propuesta formal del lenguaje Singularity}
\author[1]{Bryan Ulate}
\author[1]{Christian Rodríguez}
\author[1]{Gabriel Gálvez}
\author[1]{Jostin Álvarez}
\affil[1]{Quantum Refraction}
\renewcommand\Authands{ y }
\date{Agosto 27, 2020}

\lstset{
    escapechar={|}
}

\lstdefinelanguage{Singularity}
{
    keywords=[1]{
        begin, end, 
    },
    keywords=[2]{
        define, function, arguments, call, parameters, with, set, as, of, size, by, at,
    },
    keywords=[3]{
      read, to,
    },
    keywords=[4]{
        print,
    },
    keywords=[5]{
        while, counting, from, to,
    },
    keywordstyle=[1]\color{brown},
    keywordstyle=[2]\color{RoyalBlue},
    keywordstyle=[3]\color{ForestGreen},
    keywordstyle=[4]\color{BrickRed},
    keywordstyle=[5]\color{RoyalPurple},
    sensitive=false,
    morecomment=[l][\color{gray}]{\#},    
    morestring=[b][\color{Sepia}]",
}

\begin{document}

\maketitle

\section{Lista de tokens y su utilidad: }
%\vspace{-25pt}
%\begin{table}[H]
    \begin{longtable}{p{0.2\textwidth}p{.15\textwidth}p{.55\textwidth}}
        \toprule
        \textbf{Token} & \textbf{Regex} & \textbf{Utilidad} \\
        \midrule
            \texttt{SET} & \texttt{set} & Permite asignar el valor a una palabra \\
            \texttt{TO} & \texttt{to} &  Se utiliza en conjunto con el set para especificar el valor\\
            \texttt{AS} & \texttt{as} & Se utiliza en conjunto con el set para crear listas o matrices\\
            \texttt{LIST} & \texttt{list} & Permite creer listas\\
            \texttt{MATRIX} & \texttt{matrix} & Permite crear matrices de tamaño m x n\\
            \texttt{AT} & \texttt{at} & Permite señalar una posición en un arreglo\\
            \texttt{READ} & \texttt{read} & Permite leer una variable de la entrada estándar \\
            \texttt{PRINT} & \texttt{print} & Permite hacer una impresión en la salida estándar\\
            \texttt{IF} & \texttt{if} & Se utiliza para manejar el control de flujo\\
            \texttt{OTHERWISE} & \texttt{otherwise} & Se utiliza en control de flujo en caso de que no se cumpla la condición del if\\
            % 11
            \texttt{BEGIN} & \texttt{begin} & Abre un bloque de código. Pueden ser funciones, ciclos, condicionales, entre otros. \\
            \texttt{END} & \texttt{end} & Cierra un bloque de código. Es el opuesto al begin.\\
            \texttt{WHILE} & \texttt{while} & Indica repetición del bloque de texto siguiente\\
            \texttt{COUNTING} & \texttt{counting} & Se utiliza para decir que la variable siguiente será un índice, similar a un ciclo for \\
            \texttt{DEFINE} & \texttt{define} & Permite definir funciones o procedimientos al sucederse de la palabra \\
            \texttt{FUNCTION} & \texttt{function} & Utilizado para definir una función, al suceder la palabra \texttt{define}\\
            \texttt{DIFFERENT} & \texttt{different} & Se utiliza en conficionales para determinar si dos valores son distintos\\
            \texttt{ANSWER} & \texttt{answer} & Marca el final de una función. Permite indicar su valor de retorno \\
            \texttt{CALL} & \texttt{call} & Se utiliza para llamar una función\\
            \texttt{WITH} & \texttt{with} & Se utiliza en conjunto con la palabra \texttt{parameters} o \texttt{arguments}\\
            % 21
            \texttt{PARAMETERS} & \texttt{parameters} & Se utiliza para indicar los parámetros que se utilizarán al llamar a una función\\
            \texttt{ARGUMENTS} & \texttt{arguments} & Se utiliza para definir cuáles argumentos necesita una función al ser llamada\\
            \texttt{NOT} & \texttt{not} & Es el operador para la negación lógica\\
            \texttt{AND} & \texttt{and} & Se utiliza en los condicionales como un operador lógico \\
            \texttt{OR} & \texttt{or} & Se utiliza como el operador lógico de disyunción\\
            \texttt{XOR} & \texttt{xor} & Se utiliza como el operador lógico de o exclusivo\\
            \texttt{ADDITION} & \texttt{+} & Operador de suma \\
            \texttt{SUBSTRACTION} & \texttt{-} & Operador de resta \\
            \texttt{MULTIPLICATION} & \texttt{*} & Operador de multiplicación \\
            \texttt{DIVISION} & \texttt{/} & Operador de división \\
            % 31
            \texttt{EQUALS} & \texttt{=} & Operador de comparación \\
            \texttt{GEQ} & \texttt{>=} & Operador mayor o igual\\
            \texttt{LEQ} & \texttt{<=} & Operador menor o igual\\
            \texttt{GREATER} & \texttt{>} & Operador de mayor\\
            \texttt{GREATER} & \texttt{greater than} & Equivalente al otro operador de mayor \\
            \texttt{LESS} & \texttt{<} & Operador de menor \\
            \texttt{LESS} & \texttt{less than} & Equivalente al otro operador de menor \\
            \texttt{OPEN\_PARENTHESIS} & \texttt{(} & Utilizado para delimitar parámetros de funciones o establecer la precedencia de operaciones\\
            \texttt{CLOSE\_PARENTHESIS} & \texttt{)} & Utilizado para delimitar parámetros de funciones o establecer la precedencia de operaciones\\
            \texttt{STRING} & \texttt{["][\^{}"]*["]} & Utilizado para delimitar una hilera de caracteres. \\
            \texttt{HASH} & \texttt{\#} & Utilizado para comentarios de una línea. \\
            \texttt{INTEGER} & \texttt{[0-9]+} & Valores enteros para realizar operaciones aritméticas. \\
            \texttt{FLOAT} & \texttt{[0-9]+\textbackslash{}.[0-9]+} & Valores en punto flotante para realizar operaciones aritméticas. \\
            \texttt{IDENTIFIER} & \texttt{[a-zA-Z\_]} \texttt{[a-zA-Z0-9\_]*} & Para atrapar los nombres de variables dados por el usuario. \\
        \bottomrule
        \caption{Tokens, regex que los detectan y su utilidad}
    \end{longtable}
    \label{tab:my_label}
%\end{table}



\section{Ejemplos de código}

\subsection{Código de ejemplo que imprime una matriz de asteriscos: }

\begin{lstlisting}
|\color{blue} define function| print_asterisks |\color{blue} with arguments| n
|\color{brown} begin|
    |\color{Gray} \# Recorre las filas de la matriz|
    |\color{red}while| row |\color{red}counting from| 0 |\color{red}to| n
    |\color{brown} begin|
        |\color{Gray} \# Recorre las columnas de la matriz|
        |\color{red}while| column |\color{red} counting from| 0 |\color{red}to| n
        |\color{brown} begin|
            |\color{purple} print| "*"
        |\color{brown} end|
        |\color{purple} print| "\n"
    |\color{brown} end|
|\color{brown} end|

|\color{blue} define function| start
|\color{brown} begin|
    |\color{ForestGreen} read to| n
    |\color{orange} call| print_asterisks |\color{orange} with parameters| (n)
|\color{brown} end|
\end{lstlisting}


\subsection{Código de ejemplo con un arreglo: }
\lstinputlisting[language=Singularity]{CodigoSingularity/arreglo.sin}

\subsection{Código de ejemplo con una matriz: }
\lstinputlisting[language=Singularity]{CodigoSingularity/matrix.sin}

\section{Error lógico que se detectará: }

\subsection{Recursión sin condición de parada}
Si dentro de una función se hace un llamado a sí misma, y no hay un condicional dentro de ella, sabemos que esta es una recursión sin condición de parada que se ejecutaría infinitamente. Esto es un error y podemos reportarlo como tal. Por ejemplo:

\lstinputlisting[language=Singularity]{CodigoSingularity/bad_recursion.sin}

\section{Actividades realizadas}
\begin{table}[H]
    \caption{Actividades realizadas para la primera parte del proyecto}
\begin{tabular}{ll}
    \toprule
    \textbf{Actividad} & \textbf{Miembros responsables} \\
    \midrule
        Inclusión de tokens y regex que reconocen esos tokens & Jostin, Christian, Gabriel, Bryan \\
        Resolver la compilación del código a partir de un ejemplo& Gabriel \\
        Organizar los archivos generados por Bison y Flex en una carpeta & Christian \\
        Creación de ejemplos & Bryan y Gabriel \\
        Tabla con tokens & Jostin, Christian, Gabriel, Bryan \\
        Syntax highlighting en \LaTeX & Bryan \\
    \bottomrule
\end{tabular}
\label{tab:actividades}
\end{table}


\end{document}
